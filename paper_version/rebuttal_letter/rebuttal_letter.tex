%% journalrebuttal.tex
%% Copyright 2020 Pranav Hosangadi
%
% This work may be distributed and/or modified under the
% conditions of the LaTeX Project Public License, either version 1.3
% of this license or (at your option) any later version.
% The latest version of this license is in
%   http://www.latex-project.org/lppl.txt
% and version 1.3 or later is part of all distributions of LaTeX
% version 2005/12/01 or later.
%
% This work has the LPPL maintenance status `maintained'.
% 
% The Current Maintainer of this work is Pranav Hosangadi.
%
% This work consists of the file journalrebuttal.cls and the demo
% file journalrebuttal.tex
%% 
%%
%% journalrebuttal.tex
%% Example usage for journalrebuttal.cls: 
%%    A LaTeX class to create rebuttal documents for journal
%%    reviews
%% Created: 2020-06-28
%% Author: Pranav Hosangadi (pranav.hosangadi@gmail.com)
%% https://github.com/pranavh/JournalRebuttal_LaTeX
%% Last Modified: 2020-06-29
%% Version: 1.0
%%
\documentclass[12pt]{journalrebuttal}
\title{\textcolor{blue}{Asymptotic-Preserving Discretization of Three-Dimensional Plasma Fluid Models}}
\author{Tianwei Yu, Roman Fuchs, Ralf Hiptmair}
\journal{\emph{Communications in Computational Physics}}
\manuscriptid{CICP-OA-2023-0270.R1}

\usepackage{verbatim}

%% You can define Note commands using the \ColorNote command 
%% provided in the class. 
\newcommand{\PHNote}[1]{\ColorNote{red}{PH}{#1}}


\begin{document}
%\pagenumbering{arabic}
\maketitle

We sincerely thank the reviewers for the generous and insightful comments on our work and have revised the manuscript to address their concerns. The corresponding replies for each comments are attached below.
\nextreviewer
\begin{revcomment}
    \textcolor{violet}{Can the new 3D scheme reduces to Degond's 1D scheme?}
\end{revcomment} 
    \begin{response}
    \textcolor{blue}{
    Yes. The proposed numerical scheme in 3D is a natural extension of the 1D
    the scheme proposed in $[$P.~Degond, F.~Deluzet, and D.~Savelief, \emph{Numerical approximation 
    of the Euler-Maxwell model in the quasineutral limit}, Journal of Computational Physics, 231 (4), 
    pp.~1917--1946, 2012$]$. This is justified by considering problems with
    transverse invariance (namely, variables keep constant along the x-axis and y-axis) and
    verifying that the "projected" 3D scheme coincides with the original 1D scheme at the z-axis.
    Section~4.1 is devoted to such a purpose. To better clarify this point, we add a remark
    right after the 3D full discretization: ``When applying the ``projection to one dimension"
    described in Section~4.1.1, the fully discrete 3D AP scheme (32) collapses to (31)
    discretization proposed by Degond et. al. This is not a coincidence, because
    recovering of (31) under dimensional reduction has guided our construction."
    }
\end{response}


\begin{revcomment}
    \textcolor{violet}{
    In Fig 8, Page 19, it seems the 1D results are better 
    than the 3D results by using the same meshes. Can the new 3D scheme
    preserve some good properties of the original 1D scheme?
    }
\end{revcomment}
\begin{response}
    \textcolor{blue}{
    Technically speaking, Fig.~8 is a ``verification" rather than a ``comparison".
    The result produced by the 3D scheme, as illustrated in Section~5.1, is the
    solution along the z-axis for the cylindrical setting in Fig.~2. To better
    clarify this point, we add some comments in the description of Fig.~8:
    ``... The 3D result is generated under the setting in Fig.~2
    and restricted to the z-axis. The results are slightly different from
    the 1D result probably due to the impact of the outer boundary of the
    cylindrical domain."
    }
\end{response}


\begin{revcomment}
    \textcolor{violet}{
    In numerical tests, the mass ratio and the collision parameter 
    are computed by a constant number. Is it sufficient to
    demonstrate the good performance of the new scheme?
    }
\end{revcomment}
\begin{response}
    \textcolor{blue}{
    We agree that the robustness with respect to the mass ratio and the collision parameter
    is worth studying in dealing with the plasma model.
    However, the objective of the proposed scheme is stability with respect to
    the Debye length $\lambda$. The impacts of the mass ratio and the collision parameter
    are not of our central interest.
    In fact, our scheme does not possess uniform stability over the mass ratio $\varepsilon_*$
    yet has it over the collision parameter $\alpha^{\text{coll}}$. This can be
    inferred from (32j) by noticing the explicit term
    $\varepsilon_*^{-2}q_*(n\mathbf{u})_*^m \times \mathbf{B}^m$
    and the implicit terms
    $\varepsilon_*^{-2}\mathbf{R}^{\text{coll}, m+1}_*, \varepsilon_*^{-2}Q^{\text{coll}, m+1}_*$.
    It seems possible to achieve stability over the mass ratio $\varepsilon_*$
    by replacing $\varepsilon_*^{-2}q_*(n\mathbf{u})_*^m \times \mathbf{B}^m$
    by $\varepsilon_*^{-2}q_*(n\mathbf{u})_*^{m+1} \times \mathbf{B}^{m}$.
    However, this would significantly change the time-stepping procedure.
    We decide not to pursue this and want to be aligned with Degond's scheme.
    }
\end{response}


\begin{revcomment}
    \textcolor{violet}{
    Include the following references:
    [Fast-Converging and Asymptotic-Preserving Simulation of Frequency Domain Thermoreflectance, 
    Jia Liu \& Lei Wu. Commun. Comput. Phys., 34 (2023), pp. 65-93.]
    [A Coupled FEM-BEM Approach for the Solution of the Free-Boundary Axi-Symmetric Plasma 
    Equilibrium Problem. M. Bonotto, D. Abate, P. Bettni \& F. Villone. Commun. Comput. Phys., 31 
    (2022), pp. 27-59.]
    }
\end{revcomment}
\begin{response}
    \textcolor{blue}{
    Both paper is included in the introduction.
    The first paper is added to the literature review on AP schemes
    while the second one is mentioned when the symmetry of the problem is discussed.
    }
\end{response}


\nextreviewer
\begin{revcomment}
    \textcolor{violet}{
    The literature review provided in the paper is incomplete.
    Numerous structure-preserving schemes for plasma models involving the Euler
    and Maxwell equations exist, such as the compressible ideal MHD model.
    Relevant works should include, but not be limited to:
    \begin{itemize}
    \item[1)] The positivity-preserving and divergence-free finite volume schemes and
    central discontinuous Galerkin schemes (also using the primal and dual meshes);
    see, e.g., [K. Wu, SIAM Journal on Numerical Analysis, 56(4):2124-2147, 2018]
    and [K. Wu, H. Jiang, and C.-W. Shu, SIAM Journal on Numerical Analysis, 61: 250-285, 2023].
    \item[2)] These schemes are designed by using the geometric quasilinearization framework in
    [K. Wu and C.-W. Shu, SIAM Review, 65(4):1031-1073, 2023],
    where the multicomponent MHD equations for multi-species plasma were also studied.
    \item[3)] High-order asymptotic preserving (AP) finite difference WENO schemes with
    constrained transport were also developed for MHD equations in
    [W. Chen, K. Wu, and T. Xiong, Journal of Computational Physics, 488: 112240, 2023].
    \end{itemize}
    }
\end{revcomment}
\begin{response}
    \textcolor{blue}{
    We have added a paragraph at the end of Section~2.2. A short discussion and literature
    review on the numerical methods for plasma models is presented with a focus on the
    structure-preserving property. The literature mentioned in 3) is included in the
    literature review in Section~1 when discussing AP schemes.
    }
\end{response}


\begin{revcomment}
    \textcolor{violet}{
    The author did not provide an accuracy test
    for the convergence rate of their proposed numerical method.
    An illustrative example can be found in section 6.2 of
    [K. Wu, H. Jiang, and C.-W. Shu, SIAM Journal on Numerical Analysis, 61: 250-285, 2023],
    as well as in Examples 5.1 and 5.2 in [W. Chen, K. Wu, and T. Xiong, Journal of Computational 
    Physics, 488: 112240, 2023].
    }    
\end{revcomment}
\begin{response}
    \textcolor{blue}{
    To handle this issue, we put some effort into re-devising the matrix assembly which
    now admits a fast iterative solver such that reference solutions on sufficiently refined
    meshes are possible. A new case involving a cuboid domain is tested and
    $L^1$ and $L^2$ errors at different values of $\lambda$ are reported in the same manner as 
    suggested by the second paper.
    A smooth test case seems hard to construct due to the presence of the interface of two different
    electromagnetic boundary conditions in our underlying setting depicted in Fig.~2.
    }    
\end{response}


\begin{revcomment}
    \textcolor{violet}{
    Limited Numerical Examples: All the numerical examples provided in the paper
    are either one-dimensional or three-dimensional cylindrical. To strengthen the paper's 
    contribution,
    it is advisable to include a genuinely three-dimensional example.
    This is especially relevant as one of the primary contributions of the paper lies in the
    extension of a one-dimensional scheme to three dimensions.
    }
\end{revcomment}
\begin{response}
    \textcolor{blue}{
    We include a slightly more general 3D case in Case IV. Yet we would like to mention that 
    our current framework of primal-dual mesh generation is based on the idea of
    ``extruding 2D mesh" in Sec.~4.2.1 of
    $[$Marrone, Massimiliano.\emph{Computational Aspects of the Cell Method in Electrodynamics - 
    Abstract.} Journal of Electromagnetic Waves and Applications 15 (2001): 407 - 408.$]$.
    It has not been extended yet to arbitrary geometries, which possibly takes quite some effort.
    Therefore, we have decided to defer this part to future work. Nevertheless,
    the proposed scheme can be directly applied to arbitrary domains
    as long as the primal-dual meshes are generated properly.
    }
\end{response} 

\makerule

\end{document}

